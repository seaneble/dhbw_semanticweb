\section{Semantic Movie Search Website}
Part of the given task was to make all the data described in the two previous sections available for user interaction via a website.

We started by defining the relevant functionality of the website in order to build an HTML prototype:

\begin{itemize}
\item Search field for user input
\item Button to send the query to the server
\item A subsection to display a list of results
\item If required, a view giving details on a single result
\end{itemize}

The website involves a certain amount of dynamic information, so we decided to use a small server-sided PHP script to take care of any logical decisions. We also integrated a library for SPARQL queries in order to simplify our script. The library acts as an abstraction layer and takes care of many small items such as connecting to the server, checking for errors as well as evaluating and transforming the response from the server. \textbf{citation needed}

\subsection{The Search Interface}
We constructed a simple markup which includes all items from above listing with HTML 5 and some basic CSS styles. HTML 5 takes care of a placeholder text within the search field to guide the user and also makes sure no empty queries are sent (required form field).

Each time the user makes a request to the website, a PHP script is called to action by the webserver. The script evaluates the current URI and the determines whether a query has been submitted by the user. This is decided based on the presence of the \emph{query} variable.

If the query variable is not present, the script outputs the static HTML content of the search interface with an appropriate HTTP Content-type header.

Otherwise, the value of the query variable is extracted from the URI and used to fetch data from the server. In a first step, the user input is transformed into a SPARQL query, which is then evaluated by the Fuseki server.

\subsection{Query Construction}
The SPARQL query is constructed based on the input of the user. A single query is sent to Fuseki which returns all interesting datasets based on the ontology described in figure \ref{fig:movie-ont}. A search does not necessarily result in a homogenous result set, but can contain URIs for various entities, e.g. movies and actors.\\
Our query approach includes the following steps:
\begin{description}  
\item[Extract information]
Based on the entity, the query finds all interesting properties and relations to other entities. The query checks the type of a statement's subject and concludes, based on the movie ontology and stored instance data, that it belongs e.g. to the Movie class. Then it tries to fetch other mandatory or optional relations, e.g. the actors that play in this movie (i.e. Actor class instances that match the predicate introduced in \ref{lst:actor-pred}).

Those relations depend on the subject the query found, i.e. Movie class instances have other dependent relations than Actor instances. The query therefore aggregates all intended class alternatives (Movie, Actor or Genre) and their respective predicates using the UNION statement, which provides the ability to query graph pattern alternatives \cite{w3c:sparql}.
\item[Filter data]
All found instances at least need to provide a type and a description. Those fields are used to filter the search results of the query. Therefore, the user's input is interpreted as being a regular expression to filter the description field, which can be a movie's title, an actor's name or the name of a particular genre.
\end{description}

In addition to its URI, connected information is returned based on the entity which was found.\\
A movie also shows:
\begin{itemize}
\item the movie's title
\item the URI of each genre it belongs to
\item the URI of each actor
\end{itemize}
An actor also shows:
\begin{itemize}
\item the actor's name
\item the URI of each movie he or she is actor in
\item the actor's birth date (optional)
\end{itemize}
A genre also shows:
\begin{itemize}
\item the genre's name or description
\item the URI of each movie that belongs to this genre
\end{itemize}

\subsection{Response Evaluation}
The query constructed by the above step is sent to the Fuseki server using the SPARQL abstraction layer as well.

The server returns a SPARQL result within an XML document. The abstraction layer parses this XML structure and inserts an XSL file reference. Our PHP script then sends the resulting new XML structure to the user's browser with an HTTP Content-type of \textit{application/xml}.

The stylesheet is an extensive document which transforms the SPARQL XML into an entirely different structure for the website. One of the core challenges posed to this stylesheet was a mapping of multiple rows of the SPARQL result into one actual result item. See Listing \ref{lst:sparql-result} for the initial XML.

\begin{lstlisting}[caption={XML extract returned by the server when searching for a movie},label={lst:sparql-result}]
<result>
 <binding name="uri">
  <uri>https://github.com/seaneble/dhbw_semanticweb#Titanic</uri>
 </binding>
 <binding name="desc">
  <literal>Titanic</literal>
 </binding>
 <binding name="actor">
  <literal>Leonardo DiCaprio</literal>
 </binding>
 [...]
</result>
<result>
 <binding name="uri">
  <uri>https://github.com/seaneble/dhbw_semanticweb#Titanic</uri>
 </binding>
 <binding name="desc">
  <literal>Titanic</literal>
 </binding>
 <binding name="actor">
  <literal>Kate Winslet</literal>
 </binding>
 [...]
</result>
\end{lstlisting}

There are two rows in the result for the same movie, because two different actors are starring in it. We wanted the HTML result to look like one movie entry with a list of actors.

The XSLT loops through all the result and evaluates each result's context to see whether there are previous or following elements of the same type. They are grouped together to form an HTML structure like in Listing \ref{lst:sparql-html}.

\begin{lstlisting}[caption={HTML structure for one movie result},label={lst:sparql-html}]
<div class="result movie">
 <p><strong><a href="?query=Titanic">Titanic</a></strong> is a
 movie of the genre <em><a href="?query=Romance">Romance</a>
 </em>.</p>
 <h3>Cast</h3>
 <ul>
  <li>
   <a href="?query=Leonardo DiCaprio">Leonardo DiCaprio</a>
  </li>
 </ul>
</div>
\end{lstlisting}

Our stylesheet also adds hyperlinks to all kind of data types. The user can navigate e.g. from a genre to a movie and further on to an actor using those hyperlinks. Each hyperlink performs a new search.